\section{Linear hypothesis using linear regression}

\subsection{Data}
\begin{figure}[!ht]
  \includegraphics[width=\textwidth,height=0.4\textheight,keepaspectratio]{scatter_plot.png}
  \caption{Scatter plot of data}
  \label{fig:scatter_plot}
\end{figure}

\subsection{Implementation}
A linear function given by equation \ref{eq:simple_linear_model} was selected as the hypothesis.
Linear regression with learning learning rate($\alpha$) = 0.01 and gradient descent approach was used for optimizing the parameters.
The training stopped after reaching the maximum allowed number of iterations which was 10000.
For this specific dataset we can use a threshold of 0.5 to classify the data. The predicting data
can be assigned a class 1 (fail) if $h_{\theta}$(x) $<$ 0.5 and class 2 (pass) otherwise.

\begin{equation}
\label{eq:simple_linear_model}
Y = \theta_0 + \theta_1 * X
\end{equation}

\subsection{Observation}
The final values of $\theta$s for a run are as follows:

% Theta 0 final value
\begin{equation}
\theta_0 = -0.15184881886181856
\end{equation}

% Theta 1 final value
\begin{equation}
\theta_1 = 0.23405702659773173
\end{equation}

\begin{figure}[!ht]
  \includegraphics[width=\textwidth,height=0.4\textheight,keepaspectratio]{regression_line_0_01.png}
  \caption{Linear regression line}
  \label{fig:regression_line}
\end{figure}

The regression line corresponding to the above $\theta$s is shown in figure \ref{fig:regression_line}

\subsection{Source Code}
\lstinputlisting[language=python]{task_1.py}
