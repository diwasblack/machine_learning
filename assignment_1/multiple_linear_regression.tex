\section{Multiple linear regression}

\subsection{Model}
\begin{equation}
\label{eq:multiple_linear_regression_model}
Y = \theta_0 + \theta_1 * X_1 + \theta_2 * X_2
\end{equation}

% Regression without scaling the predictor variables
\subsection{Regression without scaling predictor variables}
\subsubsection{Initial values of parameters}
\begin{equation}
\theta_0 = 0.6520699150884046
\end{equation}
\begin{equation}
\theta_1 = 0.9861396174652116
\end{equation}
\begin{equation}
\theta_2 = 0.9915108605252747
\end{equation}

\subsubsection{Final values of parameters}
\begin{equation}
\theta_0 = 4.311257778431243e+152
\end{equation}
\begin{equation}
\theta_1 = 1.4490623380861671e+153
\end{equation}
\begin{equation}
\theta_2 = 2.685595874504226e+154
\end{equation}

\begin{figure}[!ht]
    \includegraphics[width=\textwidth, height=.4\textheight]{output_2/cost_function_alpha_0_01_unscaled.png}
    \caption{Mean squared error when predictor variables are not scaled}
    \label{fig:mean_squared_without_scaling}
\end{figure}

% Regression after scaling the predictor variables
\subsection{Regression after scaling predictor variables}
\subsubsection{Initial values of parameters}
\begin{equation}
\theta_0 = 0.4409653770961314
\end{equation}
\begin{equation}
\theta_1 = 0.16376353557709156
\end{equation}
\begin{equation}
\theta_2 = 0.2407927690962538
\end{equation}

\subsubsection{Final values of parameters}
\begin{equation}
\theta_0 = 1.545463253882367
\end{equation}
\begin{equation}
\theta_1 = 3.8202281258797064
\end{equation}
\begin{equation}
\theta_2 = 1.0008683090251478
\end{equation}

\begin{figure}[!ht]
    \includegraphics[width=\textwidth, height=.4\textheight]{output_2/cost_function_alpha_0_01_scaled.png}
    \caption{Mean squared error when predictor variables are scaled}
    \label{fig:mean_squared_with_scaling}
\end{figure}

\subsection{Observation}
Equation \ref{eq:multiple_linear_regression_model} was used to model the prediction variable Y as a linear model of predicting variables $X_1$ and $X_2$. At first, multiple linear regression without parameter scaling was used to obtain the prediction line. The cost function value kept increasing and it failed to converge as shown in the figure \ref{fig:mean_squared_without_scaling}.

When parameter scaling was used with multiple linear regression the gradient descent algorithm converged and was able to obtain the optimum values for the $\theta$s. The corresponding graph for cost function when predicting parameters were scaled is shown in figure \ref{fig:mean_squared_with_scaling}.
\subsection{Conclusion}
The data that was used with multiple linear regression shows that the predictor variable $X_2$ had higher magnitude than the values of predictor variable $X_1$. Due to the dissimilarity in magnitude of predictor variables the gradient descent algorithm when used without feature scaling diverges and fails to yield an optimum regression line.

\subsection{Source Code}

\lstinputlisting[language=python]{assignment_b.py}
